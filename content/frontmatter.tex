% !TEX root = ms.tex

\chapter*{License}
\vspace{4cm}
\lipsum[1]

\chapter*{Acknowledgements}
\lipsum[2]



\chapter*{Abstract}
\textbf{[En proceso]} Los ultimos lanzamientos de Modelos Grandes de Lenguaje
(LLM por sus siglas en ingles), da paso a nuevas implementaciones de Inteligencias
Artificiales Generalistas (AGI). Mas allá the optimizar modelos de machine learning,
el desarrollo de AGI's requiere de aplicaciones cognitivas que habiliten a las LLM
a operar de forma efectiva en aplicaciones del mundo real.

Esta memoria intruduce la Máquina de Congición-Ejecución(MCE) como un marco teórico
que descompone el diseño de un AGI como un problema the aproximacion definido por 3
variables clave: El \textit{espacio de ejecución}, textit{espacio de cognición} y 
\textit{espacio de ejecución}. Como aproximación a este marco teórico se propone
\textit{[SystemName]} como un entorno de desarrollo para probar y diseñar AGI's
implementando diferentes aproximaciones en multiples capas. Finalmente se implementan
multiples técnicas incluyendo \textit{Prompt Engineering}, codigo \textit{Exelent},
o \textit{Auto-Entrenamiento} con el fin de construir un AGI que soporte la interación
usuario-IA en un entorno no controlado.

\todo{Incluir en el abstract mencion a la implementacion en RaspBerry,
y resultados}


\chapter*{Acronyms}
\begin{itemize}
    \item \textbf{AI}: Artificial Intelligence
    \item \textbf{AP}: Agent Protocol
    \item \textbf{$A_1$}: Execution Layer Algorithm
    \item \textbf{$A_2$}: Cognition Layer Algorithm
    \item \textbf{LLM}: Large Language Model
    \item \textbf{ECM}: Execution-Cognition Machine
    \item \textbf{AGI}: Artificial General Intelligence
    \item \textbf{RAG}: Retrieval Augmented Generation 
    \item \textbf{PMPA}: Profile-Memory-Plan-Action

\end{itemize}

\begingroup
\setlength{\parskip}{0pt}
\setlength{\parindent}{3pt}
\tableofcontents
\endgroup

